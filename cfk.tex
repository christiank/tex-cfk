%
% cfk.tex -- A simple look and feel for LaTeX documents
% Christian Koch <cfkoch@sdf.lonestar.org>
%
% Just copy and include this file in your next LaTeX project: %
% cfk.tex -- A simple look and feel for LaTeX documents
% Christian Koch <cfkoch@sdf.lonestar.org>
%
% Just copy and include this file in your next LaTeX project: %
% cfk.tex -- A simple look and feel for LaTeX documents
% Christian Koch <cfkoch@sdf.lonestar.org>
%
% Just copy and include this file in your next LaTeX project: %
% cfk.tex -- A simple look and feel for LaTeX documents
% Christian Koch <cfkoch@sdf.lonestar.org>
%
% Just copy and include this file in your next LaTeX project: \include{cfk}
% Set the documentclass to ``article,'' then call \cfk in the preamble.
% That's it!
%

% Performs all sorts of setup. Call this in the preamble.
\newcommand{\cfk}[0]{
  \usepackage[paperwidth=8.5in, paperheight=11in, margin=1in]{geometry}
  \usepackage{framed}
  \renewcommand*{\familydefault}{\sfdefault}
  \pagestyle{myheadings}
  \normalsize
  \parindent 0in
  \parskip 12pt
}

% Takes three arguments: The title, the author and the date.
\newcommand{\cfktitle}[3]{
  {\Huge \bf #1}

  #2\\
  #3\\

  \hrule
}

% Corresponds to HTML's ``H1.''
\newcommand{\cfkone}[1]{{\LARGE \bf #1}}

% Corresponds to HTML's ``H2.''
\newcommand{\cfktwo}[1]{{\Large \bf #1}}

% Corresponds to HTML's ``H3.''
\newcommand{\cfkthree}[1]{{\large \bf #1}}

% Set the documentclass to ``article,'' then call \cfk in the preamble.
% That's it!
%

% Performs all sorts of setup. Call this in the preamble.
\newcommand{\cfk}[0]{
  \usepackage[paperwidth=8.5in, paperheight=11in, margin=1in]{geometry}
  \usepackage{framed}
  \renewcommand*{\familydefault}{\sfdefault}
  \pagestyle{myheadings}
  \normalsize
  \parindent 0in
  \parskip 12pt
}

% Takes three arguments: The title, the author and the date.
\newcommand{\cfktitle}[3]{
  {\Huge \bf #1}

  #2\\
  #3\\

  \hrule
}

% Corresponds to HTML's ``H1.''
\newcommand{\cfkone}[1]{{\LARGE \bf #1}}

% Corresponds to HTML's ``H2.''
\newcommand{\cfktwo}[1]{{\Large \bf #1}}

% Corresponds to HTML's ``H3.''
\newcommand{\cfkthree}[1]{{\large \bf #1}}

% Set the documentclass to ``article,'' then call \cfk in the preamble.
% That's it!
%

% Performs all sorts of setup. Call this in the preamble.
\newcommand{\cfk}[0]{
  \usepackage[paperwidth=8.5in, paperheight=11in, margin=1in]{geometry}
  \usepackage{framed}
  \renewcommand*{\familydefault}{\sfdefault}
  \pagestyle{myheadings}
  \normalsize
  \parindent 0in
  \parskip 12pt
}

% Takes three arguments: The title, the author and the date.
\newcommand{\cfktitle}[3]{
  {\Huge \bf #1}

  #2\\
  #3\\

  \hrule
}

% Corresponds to HTML's ``H1.''
\newcommand{\cfkone}[1]{{\LARGE \bf #1}}

% Corresponds to HTML's ``H2.''
\newcommand{\cfktwo}[1]{{\Large \bf #1}}

% Corresponds to HTML's ``H3.''
\newcommand{\cfkthree}[1]{{\large \bf #1}}

% Set the documentclass to ``article,'' then call \cfk in the preamble.
% That's it!
%

% Performs all sorts of setup. Call this in the preamble.
\newcommand{\cfk}[0]{
  \usepackage[paperwidth=8.5in, paperheight=11in, margin=1in]{geometry}
  \usepackage{framed}
  \renewcommand*{\familydefault}{\sfdefault}

  % If you don't want a page count:
  \pagestyle{myheadings} 

  % Otherwise:
  %\usepackage{fancyhdr}
  %\usepackage{lastpage}
  %\pagestyle{fancy}
  %\renewcommand{\headrulewidth}{0pt}
  %\fancyhead[EOR]{\thepage/\pageref{LastPage}}
  %\fancyfoot[EOLRC]{}

  \normalsize
  \parindent 0in
  \parskip 12pt
}

% Takes three arguments: The title, the author and the date.
\newcommand{\cfktitle}[3]{
  {\Huge \bf #1}

  #2\\
  #3\\

  \hrule
}

% Corresponds to HTML's ``H1.''
\newcommand{\cfkone}[1]{{\LARGE \bf #1}}

% Corresponds to HTML's ``H2.''
\newcommand{\cfktwo}[1]{{\Large \bf #1}}

% Corresponds to HTML's ``H3.''
\newcommand{\cfkthree}[1]{{\large \bf #1}}
